% Template for ISBI paper; to be used with:
%          spconf.sty  - ICASSP/ICIP LaTeX style file, and
%          IEEEbib.bst - IEEE bibliography style file.
% --------------------------------------------------------------------------
\documentclass{article}
\usepackage{spconf,amsmath,graphicx}

% It's fine to compress itemized lists if you used them in the
% manuscript
\usepackage{enumitem}
\setlist{nosep, leftmargin=14pt}

\usepackage{mwe} % to get dummy images

% Example definitions.
% --------------------
\def\x{{\mathbf x}}
\def\L{{\cal L}}

% Title.
% ------
\title{Simultaneous Segmentation of Blood Vessels, Optic Disc and Exudates in Fundal Images using Deep Neural Networks}
%
% Single address.
% ---------------
%\name{Sunil Kumar Vengalil}
%\address{International Institute of Information Technology}
%
% For example:
% ------------
%\address{School\\
%	Department\\
%	Address}
%
% Two addresses (uncomment and modify for two-address case).
% ----------------------------------------------------------
%\twoauthors
%  {A. Author-one, B. Author-two\sthanks{Some author footnote.}}
%	{School A-B\\
%	Department A-B\\
%	Address A-B}
%  {C. Author-three, D. Author-four\sthanks{The fourth author performed the work
%	while at ...}}
%	{School C-D\\
%	Department C-D\\
%	Address C-D}
%
% More than two addresses
% -----------------------
% \name{Author Name$^{\star \dagger}$ \qquad Author Name$^{\star}$ \qquad Author Name$^{\dagger}$}
%
% \address{$^{\star}$ Affiliation Number One \\
%     $^{\dagger}$}Affiliation Number Two
%
\begin{document}
%\ninept
%
\maketitle
%
\begin{abstract}
Fundal imaging is the most commonly used non-invasive technique for early detection of many retinal diseases like diabetic retinopathy.
An initial step in automatic processing of fundal images for detecting diseases is to identify the various landmark regions in the retinal image.
The most important structures visible in a  fundal image are the optic disc, blood vessels and fovea.
In addition to these, various abnormalities like exudates, micro-aneurysm and haemorrhages that help in pathological analysis is visible in fundal images.
In this work, we propose a multi-tasking deep learning architecture for segmenting optic disc and blood vessels, fovea and exudates simultaneously.
Our experimental results show that combined predictions of all these structures simultaenously gives in significant improvement in the prediction of each structures.

We achieved an F1 score of 0.78 for blood vessels segmentation, on the DRIVE test dataset, with multi-tasking using a simple U-Net architecture.
When the same architecture is used without multi-tasking, i.e. for blood vessel segmentation alone, the F1 score was 0.72 which is significantly less compared to the multi-tasking model performance.
Similarly for optic disc segmentation, we obtained an F1 score of 0.76 using multi-tasking which is a significant improvement over 0.7 without multi-tasking.
We also performed experiments on HRF dataset.
We got an F1 score of 0.79 with multi-tasking and 0.78 without multi-tasking on HRF dataset for blood vessel segmentation.
Using a more complex architecture, like Laddernet, it is possible to further improve the results.

\end{abstract}
%
\begin{keywords}
One, two, three, four, five
\end{keywords}
%
\section{Introduction}
\label{sec:intro}

These guidelines include complete descriptions of the fonts, spacing, and
related information for producing your proceedings manuscripts.

\section{Formatting your paper}
\label{sec:format}

All printed material, including text, illustrations, and charts, must be kept
within a print area of 7 inches (178 mm) wide by 9 inches (229 mm) high. Do
not write or print anything outside the print area. The top margin must be 1
inch (25 mm), except for the title page, and the left margin must be 0.75 inch
(19 mm).  All {\it text} must be in a two-column format. Columns are to be 3.39
inches (86 mm) wide, with a 0.24 inch (6 mm) space between them. Text must be
fully justified.

\section{Page title section}
\label{sec:pagestyle}

The paper title (on the first page) should begin 1.38 inches (35 mm) from the
top edge of the page, centered, completely capitalized, and in Times 14-point,
boldface type.  The authors' name(s) and affiliation(s) appear below the title
in capital and lower case letters.  Papers with multiple authors and
affiliations may require two or more lines for this information.

\section{Type-style and fonts}
\label{sec:typestyle}

To achieve the best rendering both in the printed and digital proceedings, we
strongly encourage you to use Times-Roman font.  In addition, this will give
the proceedings a more uniform look.  Use a font that is no smaller than nine
point type throughout the paper, including figure captions.

In nine point type font, capital letters are 2 mm high.  If you use the
smallest point size, there should be no more than 3.2 lines/cm (8 lines/inch)
vertically.  This is a minimum spacing; 2.75 lines/cm (7 lines/inch) will make
the paper much more readable.  Larger type sizes require correspondingly larger
vertical spacing.  Please do not double-space your paper.  True-Type 1 fonts
are preferred.

The first paragraph in each section should not be indented, but all the
following paragraphs within the section should be indented as these paragraphs
demonstrate.

\section{Major headings}
\label{sec:majhead}

Major headings, for example, "1. Introduction", should appear in all capital
letters, bold face if possible, centered in the column, with one blank line
before, and one blank line after. Use a period (".") after the heading number,
not a colon.

\subsection{Subheadings}
\label{ssec:subhead}

Subheadings should appear in lower case (initial word capitalized) in
boldface.  They should start at the left margin on a separate line.
 
\subsubsection{Sub-subheadings}
\label{sssec:subsubhead}

Sub-subheadings, as in this paragraph, are discouraged. However, if you
must use them, they should appear in lower case (initial word
capitalized) and start at the left margin on a separate line, with paragraph
text beginning on the following line.  They should be in italics.

\section{Printing your paper}
\label{sec:print}

Print your properly formatted text on high-quality, 8.5 x 11-inch white printer
paper. A4 paper is also acceptable, but please leave the extra 0.5 inch (12 mm)
empty at the BOTTOM of the page and follow the top and left margins as
specified.  If the last page of your paper is only partially filled, arrange
the columns so that they are evenly balanced if possible, rather than having
one long column.

In \LaTeX, to start a new column (but not a new page) and help balance the
last-page column lengths, you can use the command ``$\backslash$pagebreak'' as
demonstrated on this page (see the \LaTeX\ source below).

\section{Page numbering}
\label{sec:page}

Please do {\bf not} paginate your paper.  Page numbers, session numbers, and
conference identification will be inserted when the paper is included in the
proceedings.

\section{Illustrations, graphs, and photographs}
\label{sec:illust}

Illustrations must appear within the designated margins.  They may span the two
columns.  If possible, position illustrations at the top of columns, rather
than in the middle or at the bottom.  Caption and number every illustration.
All halftone illustrations must be clear black and white prints.  If you use
color, make sure that the color figures are clear when printed on a black-only
printer.

Since there are many ways, often incompatible, of including images (e.g., with
experimental results) in a \LaTeX\ document, below is an example of how to do
this \cite{Lamp86}.

% Below is an example of how to insert images. Delete the ``\vspace'' line,
% uncomment the preceding line ``\centerline...'' and replace ``imageX.ps''
% with a suitable PostScript file name.
% -------------------------------------------------------------------------
\begin{figure}[htb]

\begin{minipage}[b]{1.0\linewidth}
  \centering
  \centerline{\includegraphics[width=8.5cm]{example-image}}
%  \vspace{2.0cm}
  \centerline{(a) Result 1}\medskip
\end{minipage}
%
\begin{minipage}[b]{.48\linewidth}
  \centering
  \centerline{\includegraphics[width=4.0cm]{example-image}}
%  \vspace{1.5cm}
  \centerline{(b) Results 3}\medskip
\end{minipage}
\hfill
\begin{minipage}[b]{0.48\linewidth}
  \centering
  \centerline{\includegraphics[width=4.0cm]{example-image}}
%  \vspace{1.5cm}
  \centerline{(c) Result 4}\medskip
\end{minipage}
%
\caption{Example of placing a figure with experimental results.}
\label{fig:res}
%
\end{figure}

% To start a new column (but not a new page) and help balance the last-page
% column length use \vfill\pagebreak.
% -------------------------------------------------------------------------
\vfill
\pagebreak


\section{Footnotes}
\label{sec:foot}

Use footnotes sparingly (or not at all!) and place them at the bottom of the
column on the page on which they are referenced. Use Times 9-point type,
single-spaced. To help your readers, avoid using footnotes altogether and
include necessary peripheral observations in the text (within parentheses, if
you prefer, as in this sentence).


\section{Copyright forms}
\label{sec:copyright}

You must include your fully completed, signed IEEE copyright release form when
you submit your paper. We {\bf must} have this form before your paper can be
published in the proceedings.  The copyright form is available as a Word file,
a PDF file, and an HTML file. You can also use the form sent with your author
kit.

\section{Referencing}
\label{sec:ref}

List and number all bibliographical references at the end of the
paper.  The references can be numbered in alphabetic order or in order
of appearance in the document.  When referring to them in the text,
type the corresponding reference number in square brackets as shown at
the end of this sentence \cite{C2}.

\section{Compliance with ethical standards}
\label{sec:ethics}

IEEE-ISBI supports the standard requirements on the use of animal and
human subjects for scientific and biomedical research. For all IEEE
ISBI papers reporting data from studies involving human and/or
animal subjects, formal review and approval, or formal review and
waiver, by an appropriate institutional review board or ethics
committee is required and should be stated in the papers. For those
investigators whose Institutions do not have formal ethics review
committees, the principles  outlined in the Helsinki Declaration of
1975, as revised in 2000, should be followed.

Reporting on compliance with ethical standards is required
(irrespective of whether ethical approval was needed for the study) in
the paper. Authors are responsible for correctness of the statements
provided in the manuscript. Examples of appropriate statements
include:
\begin{itemize}
  \item ``This is a numerical simulation study for which no ethical
    approval was required.'' 
  \item ``This research study was conducted retrospectively using
    human subject data made available in open access by (Source
    information). Ethical approval was not required as confirmed by
    the license attached with the open access data.''
    \item ``This study was performed in line with the principles of
      the Declaration of Helsinki. Approval was granted by the Ethics
      Committee of University B (Date.../No. ...).''
\end{itemize}


\section{Acknowledgments}
\label{sec:acknowledgments}

IEEE-ISBI supports the disclosure of financial support for the project
as well as any financial and personal relationships of the author that
could create even the appearance of bias in the published work. The
authors must disclose any agency or individual that provided financial
support for the work as well as any personal or financial or
employment relationship between any author and the sources of
financial support for the work.

Other types of acknowledgements can also be listed in this section.

Reporting on real or potential conflicts of interests, or the absence
thereof, is required in the paper. Authors are responsible for
correctness of the statements provided in the manuscript. Examples of
appropriate statements include:
\begin{itemize}
  \item ``No funding was received for conducting this study. The
    authors have no relevant financial or non-financial interests to
    disclose.'' 
  \item ``This work was supported by […] (Grant numbers) and
    […]. Author X has served on advisory boards for Company Y.'' 
  \item ``Author X is partially funded by Y. Author Z is a Founder and
    Director for Company C.''
\end{itemize}

% References should be produced using the bibtex program from suitable
% BiBTeX files (here: strings, refs, manuals). The IEEEbib.bst bibliography
% style file from IEEE produces unsorted bibliography list.
% ------------------------------------------------------------------------- 
\bibliographystyle{IEEEbib}
\bibliography{strings,refs}

\end{document}
